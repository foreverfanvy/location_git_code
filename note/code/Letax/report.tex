
\usepackage{amsmath}
\documentclass{article}
\usepackage[UTF8]{ctex}

\title{Gravitational wave detection}
\author{Weiyi Sun,Rong Zeng,Yuduo Chen,Ziyu Shen,Jialing Wu,Yinna Zhou,YIjie Li}
\date{April 2024}

\begin{document}
\section{Executive Summary}
\subsection{background}
\paragraph{引力波研究可以追溯到20世纪初, 但到近十年,引力波探测的发展才取得了重大突破。在爱因斯坦预测引力波的存在后,许多科学家开始为探测引力波而努力,并在这过程中,许多更精密的仪器应运而生。直到2016年,人们通过双星黑洞合并首次直接观测引力波。值得注意的是,使用激光干涉的探测器主要探测高频引力波,而宇宙本身就具有一种探测工具——毫秒脉冲星,可以用于探测低频引力波。人们普遍认为,对引力波的研究可以揭示更多宇宙的信息。}
\subsection{objective}
\paragraph{了解引力波探测的原理,及引力波信号与非事件背景噪声的区别}
\subsection{methods}
\paragraph{引力波探测的核心原理是基于干涉测量技术。一束激光从激光仪中射出,经过45°倾斜放置的分光镜,分成两束相位完全相同的激光。若无引力波经过探测,则光束行进距离完全相同,探测器无法探测出信号。若有引力波经过探测,引起周围时空发生扰动,两条光束行进的距离产生细微差异,则探测器上显示的光线强度就会明显变化。信号处理时,主要运用匹配过滤算法:}
\paragraph{$(s∣ℎ)(τ)=∫−∞∞s(t)ℎ(t−τ)dt(s∣h)(τ)=∫−∞∞ s(t)h(t−τ)dt$}
\paragraph{ 然后采用信噪比计算等统计方法来决定相关峰的显著性。同时,数据处理时,采用卷积神经网络(CNNs)将数据段进行分类,并运用粒子群算法和PSO算法寻找最优解。} 
\subsection{result}
\paragraph{实验中识别出了弱信号,并消除了数据噪声}
\subsection{conclusion}
\paragraph{信号过滤是发现引力波的关键因素,通过使用匹配引力波信号和不同噪声源的特定特殊特征的滤波程序,可以提高引力波探测的可探测性和可靠性。通过使用机器学习算法,可以提高引力波探测的自动化和精度。}
\subsection{significance}
\paragraph{本论文中我们所研究的引力波具有十分广泛且重要的作用。首先,对引力波的直接探测可以验证广义相对论,揭示新的物理理论。其次,引力波提供了一种全新的观测宇宙的手段,我们可以通过探测引力波获取早期宇宙的信息。同时,引力波与电磁波天文学的的结合,使得我们可以从多个角度探测天体事件。再而,通过分析引力波信号,我们可以研究黑洞与中子星合并时的性质。}
\section{Introduction}
\paragraph{我们对宇宙的感觉主要是由电磁波提供的。每一个电磁波谱的打开,都会为我们带来前所未有的发现。利用伽马射线,X射线,紫外和红外的观测,我们也取得了类似的进展,给天文带来了新的认识。如果电磁天文学给了我们看到宇宙的眼睛,那么引力波天文学为我们开启了宇宙深处未知奥秘的研究之门。引力波的发现开辟了引力波天文学研究的新纪元,是继以电磁辐射为探测手段的传统天文学之后,人类观测宇宙的一个新窗口。1916年,爱因斯坦基于广义相对论预言了引力波的存在。引力波,在物理学中是指时空弯曲中的涟漪,通过波的形式从辐射源向外传播,这种波以引力辐射的形式传输能量。换句话说,引力波是物质和能量的剧烈运动和变化所产生的一种物质波。然而由于引力波的强度为物体运动速度与光速之比的五次方,其强度极小,这导致引力波的探测难度极大。}
\paragraph{直到2016年2月11日激光干涉引力波天文台 (LIGO)对外宣布其直接观测到引力波信号,从此开启了引力波天文学、引力波物理学以及量子–宇宙物理研究的新纪元。这一实验结果不仅是对爱因斯坦创立广义相对论所预言的引力波的一次直接验证,更为人类进一步探索宇宙的起源、形成和演化提供了一个全新的观测手段。路易斯安那州立大学物理与天文学教授冈萨雷(Gabriela González)所说:“这项探测是一个是时代的开始:引力波天文学研究领域现在终于不再是纸上谈兵”。引力波的探测还具有重大的战略意义,通过开展空间引力波探测,可全面推动中国空间高精度引力参考传感器、星间超高精度激光干涉测量、高精度卫星编队、卫星姿轨控和温控等技术的成熟,带动一系列对国民经济和国家战略需求有重要价值的关键技术的发展。\textbf{本文主要介绍引力波探测原理和作用以及相关算法,同时对关于引力波数据处理中的噪声消除进行了较为详细的介绍。}}
\section{引力波探测基础原理}
\paragraph{激光干涉引力波天文台(LIGO)在2015年9月14日成功探测到了双黑洞并合产生的引力波。在此之前,人类观测宇宙的手段一直依赖于电磁波,这包括可见光、射电波、红外线等等。然而,引力波探测的成功标志着我们获得了一种全新的观测能力。通过直接探测引力波,我们得以窥探宇宙中发生的重大事件,如黑洞合并、中子星碰撞等,这些是传统电磁波无法捕捉到的现象。}
\paragraph{引力波探测的核心原理是基于干涉测量技术。在光学干涉仪中,激光束被分成两条垂直的光路传播,然后再次合并在一起。当其中一条光路受到引力波的影响时,光束的长度会微微改变。这微小的长度变化通过干涉测量技术被探测,从而实现对引力波的探测。}
\paragraph{LIGO的探测就是采用了激光干涉的原理。一束激光从激光仪中发出,经过一面45°倾斜放置的分光镜,分成两束相位完全相同的激光,并向互相垂直的两个方向传播。这两束光线到达距离相等的两个反射镜后,沿原路反射回来并发生干涉。如果光束行进的距离完全相同,它们的光波将完美错开,发生完全破坏性干涉,此时探测器上是探测不到激光信号的。当有引力波经过探测时,使探测器周围的空间发生扰动,导致空间本身在一个方向上拉伸,同时在另一个方向上压缩,两束激光束走过的路程就会产生细微的差异,相位发生交错,探测器上的光线强度就会发生明显的变化。
利用激光干涉仪测量引力波的基本想法很简单。引力波带来的时空扭曲,会引起干涉仪两臂相对长度的微小变化。这个微小变化反映在干涉仪中返回分束镜时两个光场的相位差异上,于是产生一个与引力波对应的光学干涉信号。但是引力波带来的时空扭曲非常微小,为获得足够的灵敏度,高新 LIGO 探测器在光学上采取了数个增强措施。首先,每个干涉 臂本身是法布里—珀罗谐振腔。 进入干涉臂的激光,不像一般迈克尔孙干涉仪,只走一个来回就 回到分束镜与另一臂的光束发生干涉,而是被“保存”在干涉臂中一段时间。激光在法布里-珀罗谐振腔中来回振荡约 300 次才 返回分束镜,意味着光场的相位 对长度变化的敏感度得到了 300 倍的增强。或者换句话说,干涉仪的有效臂长提高了 300 倍。与 此同时,腔中激光功率也因为共 振增强作用,提高了 300 倍。其 次,干涉仪工作在暗态附近,大部分激光被反射 回原激光方向。因此,在激光之后、分束镜之前,放置了一个部分透过的镜子,用以共振增强 干涉仪中的激光,增强效果约35倍。于是,20 W 的激光输入,在分束镜处增强为约700 W,在每个臂中增强为100 kW。另外,在干涉仪的输出端 也放置了一个部分透过的镜子,用于增强信号 提取和探测器带宽。}
\paragraph{高新LIGO探测器能够探测极微弱的引力波应变,必然对环境中的微小振动非常敏感。它本质上是一个巨大的地震仪,可以感应附近路上的 车辆,远处海浪拍岸,还可感应地球上几乎所有重要的地震。这些机械振动是低频端的主要 噪声来源。为降低震动噪声,每个镜子都被悬 挂在一个复杂的四级摆系统上。}
\section{引力波的发展历史}
引力波研究的历史可以追溯到20世纪初,但由于探测设备的限制,直到20世纪末21世纪初才出现真正的突破。
早在1916年,爱因斯坦就根据广义相对论预言了引力波的存在,而经典力学的牛顿万有引力定律无法解释引力波的存在。在接下来的几十年里,关于引力波的争论和研究不断推进。 在实验方面,被称为直接探测引力波第一人的约瑟夫·韦伯声称使用现在被称为韦伯棒的探测器探测到了第一个引力波。 然而,考虑到到 20 世纪 70 年代中期,世界各地其他研究小组进行的重复实验未能发现任何信号,而到 1970 年代末,对韦伯结果的怀疑已成为共识,因此对他的观察的有效性的怀疑有所增加。 。理论上,1974年赫尔斯-泰勒双星脉冲星的发现为引力波的存在提供了第一个间接证据。 作为人类发现的第一个双脉冲星系统,PSR B1913+16是对爱因斯坦广义相对论的重要验证。受之前所有研究的推动,一些小组尝试使用激光干涉仪来探测引力波,这导致了 GEO600、LIGO 和 Virgo 等更加灵敏的仪器的出现。 
2016年2月11日,LIGO-Virgo合作宣布于2015年9月14日首次直接观测到源于双黑洞合并的引力波。这一事件标志着人类引力波探测进入了新时代。 次年,即2017年,诺贝尔物理学奖再次授予引力波研究领域,雷纳·韦斯、基普·索恩和巴里·巴里什因对引力波探测的贡献而获奖。 

\section{引力波的作用}
\paragraph{引力波的作用和意义非常广泛,以下是一些主要方面:}
\paragraph{1.验证广义相对论:引力波的直接探测验证了广义相对论的关键预言,增强了我们对宇宙和重力工作方式的理解。}
\paragraph{2.新的物理学:引力波探测还可能揭示物理学中的新现象或新理论,例如探测到与预期不符的信号可能会提示新的物理过程或粒子的存在。引力波的探测对于研究引力场的波动性、量子性具有不可替代的作用。同时,通过引力波,人们可以研究强引力场物理,在强引力场中精确检验广义相对论,以及区分不同的引力理论。}
\paragraph{3.宇宙学研究:引力波提供了一种全新的观测宇宙的手段。理论上,大爆炸之后不久产生的引力波可能仍然在宇宙中传播。如果能够探测到这些引力波,它们可能会提供关于宇宙早期状态的珍贵信息。由于引力波大多产生于极致密的天体的剧烈运动,或者极早期宇宙的演化,而且引力波一旦产生,当其在宇宙中传播时几乎不发生任何相互作用,与电磁波(例如光波)不同,因此它可以携带关于其源头(例如黑洞或中子星合并)的未被干扰的天体物理和早期宇宙信号,除此之外,致密双星作为引力波源,通过对引力波的振幅和相位观测可以精确确定波源的光度距离。}
\paragraph{4.多信使天文学:引力波天文学与传统的电磁波天文学结合,可以形成多信使天文学,这允许我们从多个角度同时观察天体事件,从而获得更完整的事件画面。通过引力波的探测可以搜寻未知的天体和物质,它能提供电磁辐射等传统的天文观测方式所不能够获取的信息。这是人类观测宇宙的一个新窗口,扩展了天文学研究的视野和领域。结合电磁手段确定的红移信息,这一类引力波波源可以作为标准铃声来探测宇宙的膨胀历史。}
\paragraph{5.黑洞和中子星的研究:\textbf{通过分析引力波信号,科学家可以学习到黑洞和中子星合并时的性质,例如它们的质量、自旋和合并过程中释放的能量。为人类研究中子星内部结构、黑洞视界附近物理、超新星爆发过程,以及极早期宇宙演化等提供了唯一的探针。}}

\subsection{噪声的来源与特性}
\paragraph{在引力波探测领域,噪声是一个极为重要的问题,因为它直接影响着探测系统的灵敏度和可靠性。噪声主要来自于多个来源,每个来源都有其独特的特性和影响程度。}
噪声主要由内部噪声,环境噪声,宇宙背景噪声和白噪声构成。接下来将从这几个方面逐个介绍噪声
\subsubsection{内部噪声}
\paragraph{内部噪声主要来源于引力波探测器和相关设备的本底噪声,以及数据处理过程中产生的噪声。内部噪声的特性包括随机性、稳定性、频率依赖性等。首先,激光干涉测量链路中的各个环节都会引入噪声如激光产生、发射、传播、接收、处理的各个环节,包括激光器噪声、光学平台噪声、捕获与跟瞄噪声、超稳时钟噪声、超稳激光器噪声、干涉光学组件噪声、检验质量噪声、无拖曳控制噪声等。其次,干扰抑制系统也会引入噪声,主要包括直接作用于检验质量的力干扰、执行噪声以及无拖曳控制带来的耦合干扰包括力干扰、执行噪声、无拖曳控制等。最后,电磁力噪声也是引力波探测中需要考虑的重要因素。它可能包括波动磁场、直流偏置、波动电场等,这些电磁场的变化可能会对检验质量产生力的影响,进而干扰引力波信号的探测。奈奎斯特噪声与直流偏置以及洛伦兹力等也可能对系统的性能产生影响[C)1994-2023 China Academic Journal Electronic Publishing House, All rights reserved,336深空探测学报(中英文)2023年]}
\subsubsection{环境噪声}
\paragraph{环境噪声是指引力波探测器所处的环境中存在的噪声,如地震噪声、大气噪声、海洋噪声等。环境噪声对引力波信号的探测具有严重影响,需要采取相应的抗干扰措施。}
\subsubsection{宇宙背景噪声}
\paragraph{宇宙背景噪声是指宇宙空间中存在的广泛噪声源,如宇宙微波背景辐射、银河系射电噪声等。宇宙背景噪声对引力波信号的探测具有全局性影响,目前尚无完全解决宇宙背景噪声问题的方法。}
\subsubsection{白噪声}
\paragraph{白噪声是一种具有平坦功率谱的噪声,对引力波信号的探测具有较大影响。}
噪声对引力波探测结果得影响十分巨大,所以减少尽量剔除非引力波得噪声是十分有必要得事情,接下来将介绍几种概率论处理算法从而减少噪声。

\subsection{噪声的评估}
\paragraph{噪声的评估在GW天文学中意义非凡,GW天文学数据处理的主要目标便是将信号从噪声中提取出来。噪声是一种随机过程,需用概率论描述。在引力波信号与噪声的分离技术中,常用的方法包括滤波技术、统计分析和机器学习等。滤波技术通过设计特定的滤波器,将噪声频率成分滤除,从而保留引力波信号。例如,在LIGO(激光干涉引力波天文台)项目中,科学家们采用了先进的数字滤波器,成功地从背景噪声中提取出了微弱的引力波信号。统计分析方法则利用信号和噪声的统计特性进行区分,如通过计算信号的信噪比来评估信号的质量。机器学习技术近年来在引力波信号处理中展现出巨大的潜力,通过训练模型学习信号和噪声的特征,实现更高效的信号提取。
以机器学习为例,其应用不仅提高了信号识别的准确性,还降低了误报率。一项研究利用深度学习算法对引力波信号进行识别,结果显示,该算法在识别真实信号的同时,有效减少了由噪声引起的误报。此外,机器学习还可以处理复杂的噪声环境,如地球引力波探测器在地面运行时受到的各种干扰。通过训练模型适应这些干扰,机器学习算法能够更准确地提取出引力波信号。}
\subsection{抑制算法}
\textbf{匹配滤波法}
\paragraph{匹配滤波是一种常用的技术,通过不断优化信号模板的形状或参数来与数据进行匹配,以识别出特定的信号模式在平稳高斯可加背景噪声假设下, 可以得到最优线性引力波探测方法一匹配滤波法。假设$s(t)$为$LIGO$获取的应变信号,$h_{i}(t)\in T$为匹配模板,其中$T$为模板库, 则模板$h_{i}(t)$对应的匹配滤波输出为:
$z_{i}(t)=\int_{-\infty}^{\infty} \frac{\tilde{s}(f) \tilde{h}_{i}^{*}(f)}{S_{d n}(f)} \mathrm{e}^{j 2 \pi f t} \mathrm{~d}f$其中,$\tilde{s}(f)$和 $\tilde{h}_{i}^{*}(f)$分别为$s(t)$  和 $h_{i}(t)$的傅里叶变换。$S_{d n}(f)$为信号的双边功率谱, 与单边功率谱$S_{n}(t)$的关系为:
$S_{n}(f)=\left\{\begin{array}{ll}0 & , \quad f<0 \\2 S_{d n}(f), & f>0\end{array},\right.$
设$x(t)=\int_{-\infty}^{\infty} \frac{\tilde{s}(f)}{S_{n}(f)} \mathrm{e}^{j 2 \pi f t} \mathrm{~d} f$则模板 $h_{i}(t)$对应的匹配滤波输出为:$z_{i}(t)=\int x(\tau) h_{i}(t-\tau) \mathrm{d} \tauLIGO$获取的应变数据为离散时间信号$s[n]=s(n T)$,其中$T$为采样周期, 对上式进行离散化后得到:
$z_{i}[n]=T \sum_{m} x[m] h_{i}[m-n]$,其中,$h_{i}[n]=h_{i}(n T)$。则匹配滤波的输出为:
$z[n]=\max _{i}\left(z_{i}[n]\right)$.[C1994-2022 China Academic Journal Electronic Publishing House. All rights reserved. http://www.cnki.net262天文学进展40卷 应用于引力波探测的深度学习网络结构复杂度研究
马存良1,詹 超1,嘉明珍1,贺观圣2,李伟军2,易见兵1]}
\subsection{粒子群算法}
\paragraph{粒子群算法,是一种基于群体智能的优化算法。它模拟了鸟群觅食的行为,通过个体间的信息共享和协作来寻找最优解。粒子群算法的工作原理主要基于群体智能和个体间的信息共享。算法中,每个粒子都代表一个潜在的解,并在搜索空间中移动以寻找最优解。每个粒子都根据自身的经验和整个群体的经验来调整自己的速度和位置。
具体来说,每个粒子都有一个位置向量和一个速度向量,分别表示其在搜索空间中的位置和移动方向。在每次迭代中,粒子会更新自己的速度和位置,以靠近个体最优解和全局最优解。个体最优解是粒子自身在搜索过程中找到的最优解,而全局最优解则是整个群体中找到的最优解。
粒子速度和位置的更新通常使用以下公式:
速度更新公式:v[i](t+1) = w * v[i](t) + c1 * rand() * (pbest[i] - pos[i](t)) + c2 * rand() * (gbest - pos[i](t))
位置更新公式:pos[i](t+1) = pos[i](t) + v[i](t+1)
其中,v[i](t)和pos[i](t)分别表示第i个粒子在t时刻的速度和位置,w是惯性权重,c1和c2是学习因子,rand()是随机数函数,pbest[i]是第i个粒子的个体最优位置,gbest是全局最优位置。
这些公式描述了粒子如何根据个体和群体的历史经验来调整自己的速度和位置。通过多次迭代,粒子群算法能够逐渐收敛到最优解。}
\subsubsection{PSO算法}
\paragraph{粒子群优化算法(PSO)的原理主要基于模拟自然界中鸟群、鱼群等群集生物的觅食行为。在PSO中,每个粒子都代表搜索空间中的一个潜在解,它们通过自身的经验和同伴的经验来更新自己的位置和速度,从而寻找问题的最优解。
具体而言,每个粒子都有一个位置向量和一个速度向量,它们分别表示粒子在搜索空间中的位置和移动方向。在每次迭代中,粒子会根据自己的个体最优位置(pBest)和全局最优位置(gBest)来更新自己的速度和位置。
粒子群优化算法的核心公式包括速度更新公式和位置更新公式。速度更新公式决定了粒子在搜索空间中如何调整自己的移动方向,而位置更新公式则根据新的速度来确定粒子新的位置。
速度更新公式一般形式如下:
v[i](t+1) = w * v[i](t) + c1 * rand() * (pBest[i] - x[i](t)) + c2 * rand() * (gBest - x[i](t))
其中,v[i](t+1)是粒子i在下一时刻的速度,v[i](t)是粒子i在当前时刻的速度,w是惯性权重,用于控制粒子当前速度对下一时刻速度的影响;c1和c2是学习因子,分别表示粒子向个体最优位置和全局最优位置学习的权重;rand()是随机数函数,用于增加算法的随机性和多样性;pBest[i]是粒子i的个体最优位置,x[i](t)是粒子i在当前时刻的位置,gBest是全局最优位置。
位置更新公式相对简单,一般形式如下:
x[i](t+1) = x[i](t) + v[i](t+1)
即粒子在下一时刻的位置等于当前位置加上更新后的速度。
通过这些公式,粒子群优化算法能够模拟群集生物的协作行为,在搜索空间中寻找问题的最优解。同时,由于算法中引入了随机性,使得算法能够在一定程度上避免陷入局部最优解,从而具有更强的全局搜索能力。需要注意的是,PSO算法的性能和效果受到多个参数的影响,如惯性权重w、学习因子c1和c2等。}

\section{引力波信号和噪声的计算}
\subsection信号处理基础}
\subsubsection{信号的定义与特性}
\paragraph{信号即是携带在以时间或空间为变量的函数中的信息,可以用函数s(t)或S(x,y)来表示。信号可以是连续的,也可以是离散的。信号处理就是将信号中的信息提取出来。
信号是由引力波源产生的引力波信号,经引力波探测器的响应,而记录到探测器上的成分,即 $h(t)=h^+(t)F^+(\theta,\phi)+h\times(t)F\times(\theta,\phi)$。
信号具有确定性,但噪声本身具有误差起伏,通常用 $n(t)$ 来表示。探测器记录到的数据是信号和噪声两部分的线性叠加,即 $D(t)=n(t)+h(t)$。信号分为数字信号和模拟信号,模拟信号是时间的连续函数而数字信号是分立的采样序列。}
\subsubsection{傅里叶变换}
\paragraph{傅里叶变换在信号计算中有着至关重要的作用。傅里叶变换是一种将信号从时域转换到频域的数学方法,它可以将一个信号分解成不同频率的正弦和余弦波。在信号处理中,傅里叶变换被广泛应用于滤波、频谱分析、信号压缩等领域。傅里叶变换可被用于匹配滤波和卷积神经网络中,用于计算信号和模板的相似度。一般形式: 
$F(\omega) = \int_{-\infty}^{\infty}f(t)e^{-j\omega t}dt$}
\subsubsection{卷积}
\paragraph{卷积在滤波,计算PSD时有作用。设f(t)和g(t)为两个信号,它们的卷积结果为$(f * g)(t) = \int_{-\infty}^{\infty} f(\tau) \cdot g(t - \tau) , d\tau$
在引力波信号处理中,卷积可以用于合成引力波信号和噪声,以便后续进行滤波和分离。}
\subsubsection{窗函数}
\paragraph{窗函数通过将输入信号进行特征化的局部分析,从而根据不同需求优化分析结果。引入窗函数,可以减少频谱泄漏和旁瓣电平,提高信号处理的性能。在引力波信号处理中,常用的窗函数有汉宁窗和汉明窗等。
汉明窗(Hamming Window):$w(n) = 0.54 - 0.46 * cos(2 * π * n / (N - 1))$
汉宁窗(Hanning Window):$w(n) = 0.5 * (1 - cos(2 * π * n / (N - 1))$}

\subsubsection{滤波}
\paragraph{滤波是信号处理中的一种关键技术,可以用于去除噪声、提取有用信号等。在引力波研究中,滤波技术可以帮助我们从噪声中提取出信号。滤波的目的是塑造信号的频谱,可通过卷积实现。在给定物理参数的基础上,虽然可以精确描述引力波信号,但此类信号通常淹没在随机噪声之中。\textbf{因此,从噪声中分离出信号,必须依赖于统计学手段。}通过对仪器特性的掌握,可以预知噪声的统计特性;进一步地,可以据此计算出相应数据x的概率p(x)。若在数据中存在引力波信号的情况下,仍采用噪声假设进行计算,将会改变相应的概率分布。同时,若更换模型,在引力波信号存在的前提下,计算得到测量数据的概率,将得到不同的概率p(x)。从引力波探测器数据中寻找引力波信号存在的方法,可以概括为对比信号模型下的概率p(x)与噪声模型下的概率p'(x)。在对比信号模型下的概率p(x)与噪声模型下的概率p'(x)时,我们需要采用一系列的统计测试方法。这些方法能够帮助我们量化信号与噪声之间的差异,从而判断数据中是否存在引力波信号。}
\subsection{信号的估计与探测}
\paragraph{信号估计是指在给定观测数据的基础上,通过对数据进行分析,推断出信号的特性参数,如幅度、频率、相位等。信号探测则是从大量噪声背景中识别出有用信号,从而实现对信号的检测和定位。这两者密切相关,信号估计为信号探测提供了有力支持,而信号探测又能进一步优化信号估计的结果。}
\subsubsection{最大似然估计}
\paragraph{最大似然估计是一种常用的信号估计方法,其基本思想是寻找一个使得观测数据出现的概率最大的参数值。通常,引力波信号的探测问题可转化为统计学假设检验问题。零假设H0表示数据中无引力波信号,即$D()=n(t)$;替代假设H0表示数据含信号和噪声,即$D(t)=n(t)+h(t)$。在频率学派框架下,需进行假设检验来判断两个假说。数据分类到R和R'子集,R中接受零假设,否则拒绝。过程中可能犯两类错误:(误警)和type IIerror(漏警)。type Ierror概率是检验显著性,正确归类概率是检验能力。实际操作中,似然函数比A(D)在Bayesian、最小最大方法、Neyman-Pearson方法下具有较好性质。
$\Lambda(D)=\frac{P\left(D \mid H_{1}, I\right)}{P\left(D \mid H_{0}, I\right)}$
在引力波数据处理中,通常采用Neyman-Pearson的方法,该方法着重于最大化检验能力,即在给定检验的显著性水平(或误警率)的前提下,通过似然函数比来实现。首先,对于零假设 $H_0$,噪声模型的似然函数可以表示为$P(D|H_0) \propto \exp(-\frac{1}{2} \langle D|D \rangle)$,其中D是数据。由于信号与噪声是线性相加的,因此我们可以建立替代模型的似然函数,即在数据中抽出信号后,剩余的部分应符合噪声模型。替代模型的似然函数可以写为
$P(D|H_1(\theta)) \propto \exp(-\frac{1}{2} \langle D - h(\theta)|D - h(\theta) \rangle)$,其中$h(\theta)$ 
是信号模板, $ \theta $是参数。将这两个似然函数代入似然函数比 $\Lambda$的表达式中,发现其中的系数与参数无关,因此不会影响最终的探测结果。}
\subsubsection{功率谱估计}
\paragraph{功率谱估计可被用于分析引力波信号的频率成分。具体而言,通过对引力波信号进行快速傅里叶变换(FFT)并取其模平方,可以得到其功率谱。然后,可以通过计算功率谱的连通分量个数特征来识别不同类型的引力波信号。[C)1994-2023 China Academic Journal Electronic Publishing House. All rights reserved. http:/www.cnki.net第 49 卷第4期 吴珊珊,等:基于连通分量个数特征的引力波信号识别算法研究]}
功率谱估计对于高斯、稳态的时序噪声n(t),可以用功率谱密度Sn(f)或者其开根后的功率谱密度去描述噪声的频率域性质。
运用WienerKhintchine 定理,可以证明噪声的功率谱密度是其自相关函数的傅里叶变换。更进一步,也可得到 
$\left\langle\tilde{n}^{*}(f) \tilde{n}\left(f^{\prime}\right)\right\rangle=\frac{1}{2} S_{n}(f) \delta\left(f-f^{\prime}\right)$
Welch方法是一种高效的功率谱估计方法,通过将信号分成多个子段,并对每个子段进行平均,从而减小噪声的影响。这种功率谱估计算法可以应用于不同类型的信号上,包括高信噪比信号和低信噪比信号。对于高信噪比信号,可以选择矩形窗或凯撒窗,因为它们具有较高的频谱分辨率,但信号频率附近的噪声水平较高。对于低信噪比信号,可以选择汉宁窗或切比雪夫窗,因为它们对频谱泄漏抑制效果较好,信号频率附近噪声水平未受影响,但其分辨率相对较低。对于短信号,直接应用Welch功率谱估计法会导致频谱分辨率低,信号容易被噪声干扰导致探测失败。因此,短信号的功率谱估计无法直接通过Welch功率谱估计法来实现。[计算机时代2018年第2期 Welch功率谱估计中窗函数的选择与算法分析 邢晓晴,朱根民]
\section{数据处理方法}
\subsection{处理信号和数据分析}
\paragraph{引力波信号是从背景噪声中暗示或揭示引力波信号中最关键的部分。像匹配滤波这样的方法可以用于检测——它将观测数据与理论波形进行比较。不管计算量如何,匹配滤波与与大量波形相关的体积数据相关联。这一关键步骤构成了在从太空天文台接收到的数据中识别引力波的基础。}
\paragraph{信号处理技术通过克服观测数据噪声,成为引力波信号检测的中心技术。在这种情况下应用的主要算法类型是匹配过滤。}
\paragraph{$(s∣ℎ)(τ)=∫−∞∞s(t)ℎ(t−τ)dt(s∣h)(τ)=∫−∞∞ s(t)h(t−τ)dt$}
\paragraph{它通过计算每个时移的相关性来确定相关性发生时的最大时间相关性来做到这一点,这表明可能存在一个信号。然后考虑采用信噪比(信噪比)计算等统计方法,来决定相关峰的显著性。}
\subsubsection{信号滤波器的类型}
\paragraph{a)频域滤波器:这些滤波器作用在频域内,并据此创建的目标是增加/减少信号给定特定的频率分量。一些典型的例子是低通滤波器、高通滤波器、带通滤波器和陷波滤波器。在引力波检测的情况下,带通滤波器通常被用来施加只有引力波信号传输的频带,同时切断其他频率。}
\paragraph{b)时域滤波器:与频域滤波器相比,时域滤波器是对信号的时域表示执行的。移动平均滤波器、中值滤波器和高斯滤波器等技术通常用于实现噪声抑制,并同时捕获信号的重要特征,如瞬态事件或尖峰。在引力波分析中,采用时域滤波器来消除可能干扰引力信号的波检测的短暂噪声伪影或仪器故障。}
\paragraph{c)自适应滤波器:自适应滤波器能够根据输入信号和噪声环境的特性,动态地重新计算其参数。它们通常用于不平稳的信号和噪声的情况下,或当考虑到信号和噪声的时变特性时。采用自适应算法,如LMS算法或RLS算法,在保留信号特征的同时提供最优的去除噪声。}
\subsubsection{实施和优化}
\paragraph{在利用信号滤波器进行引力波检测时,需要考虑的方面包括滤波器的设计、计算效率和优化方法。创建一个有效的滤波器包括为它确定合适的参数,如截止频率、滤波器顺序和滤波器类型,所有这些都依赖于对手头数据的测量和预期的引力波波形。除此之外,优化技术,如FFT(即快速傅里叶变换)或卷积算法,有助于对来自引力波天文台的数据流进行实时或近距离分析。}
\subsubsection{结论}
\paragraph{信号过滤是发现引力波的关键因素。结果表明,我们识别出了弱信号,并消除了数据噪声。通过使用匹配引力波信号和不同噪声源的特定特殊特征的滤波程序,研究人员可以提高引力波探测的可探测性和可靠性,从而扩大我们对宇宙的认识。}
\subsection{机器学习技术}
\paragraph{引力波观测与机器学习紧密相连,为识别复杂的、噪声加密的数据形式提供了强大的工具。卷积神经网络(CNNs)是最广泛实现的技术,在时间序列数据的特征提取中非常有效。}
\subsubsection{卷积神经网络(CNNs)}
\paragraph{卷积神经网络是一种深度学习模型,专门设计用于处理网格拓扑数据结构,如图像和时间序列数据。针对引力波数据分类的cnn被专门训练为将数据段分类为包含引力波信号或噪声,因此,检测将变得更加有效。}
\paragraph{培训流程}
\paragraph{1.数据准备:训练数据集由标记的数据部分组成,每个部分标注为引力波信号段或噪声。输入特征被归一化以增强数据,而数据集被增强以带来更多的变化。}
\paragraph{2.模型架构:CNN介绍了其由多层卷积、池化和全连接层组成的架构设计。卷积神经网络具有卷积层的特征,其中特征从数据中获得,而池化层降低了特征图的复杂性。}
\paragraph{3.损失函数和优化:选择一个损失函数,如二值交叉熵来测量预测的标签和观测到的标签之间的差异。SGD或其变化,包括Adam或RMSprop,被用于优化网络参数,因此,以最小化损失函数。}
\paragraph{训练:CNN在训练数据集上进行迭代训练,通过反向传播和梯度下降来调整网络权值。在训练阶段,这个过程继续进行,直到满足收敛或预定的停止条件。}
\subsubsection{推理和分类}
\paragraph{一旦网络被训练好,它将能够非常有信心地预测看不见的数据是引力波信号还是噪声。在进行推理时,模型层会处理输入数据,并检索相关特征,然后利用学习到的模式来进行预测。
卷积神经网络是一类深度学习模型,它被组织成网格状的拓扑结构,并处理过图像或时间序列数据。cnn在引力波检测中的应用包括训练神经网络,用引力波信号和噪声来区分数据部分;因此,检测变得更有效。}
\paragraph{数据准备:这组数据由标记的片段组成,每个片段都标注为包含引力波信号或仅包含噪声。对数据进行预处理,以规范化输入的特征,并增强数据集以增加可变性。}
\paragraph{模型架构:CNN的架构现在被定义,包括卷积、池化和全连接的多层。卷积层负责从输入数据中提取空间特征,然后池化层向下样本特征图,从而降低了复杂性。}
\paragraph{损失函数和优化:所使用的损失函数是二进制交叉熵,它被选择用来测量预测的标签和正确的标签之间的差距。随机梯度下降(SGD)或其变体,如Adam或RMSprop,用于优化网络参数}
\paragraph{训练:CNN在训练数据集上进行迭代,并使用反向传播和梯度下降来改变网络权值。算法的训练达到收敛或满足停止的标准-后者首先出现。}
\paragraph{推理和分类:经过训练后,CNN可以很准确地显示包含引力波信号或噪声的数据片段。该模型执行推理,在此过程中,输入数据经过训练模型的各个层,允许提取特征,并使用学习到的模式进行预测。}
\section{report主要收获总结}
\paragraph{首先在工具使用上我们在此种学习了python和Latex两种工具,其中python对数据处理的三个库:pandas库,numpy库,matplot库进行了了解和学习。在python的使用上主要是采用了基于pycharm的IDE(尝试过vscode和Linux的方法但是在安装库文件的时候出现了各种各样的问题)而Latex则基于Linux系统(VMware的ubuntu系统)采用vscode和Sumatra PDF软件的方法进行输出pdf。}
\paragraph{而在物理概念的理解方面我们首先学习了搜索文献的途径:采用arxiv和谷歌学术的途径搜索关键词引力波,LIGO,探测原理,波噪声处理。而在文献阅读上我们采用了DEEPL翻译和txyz.ai对搜集的相关文件进行阅读,然后进行阅读,将我们完全理解不了的部分进行删除。整理得出前面的内容}
\paragraph{而最后我们基于LDC-manual的官方建议文档,对其中操作原理进行了理解:}
\paragraph{1.首先在基于曲率对应的建系方法,在原本的球坐标系中建立了一个动态变化的坐标系(官方文档6.1.1部分)来建立切平面和密切面,然后对于波模型部分我们没有办法理解,网络上面的资料没有告诉我们如何建立模型,于是我们决定忽略此部分的大量公式的原理,然后到框架模型建立的环节,我们发现这个本质和高等数学中的立体坐标系很像,两个与轴的夹角来建立坐标的表示方法。紧接着他们利用极化的方法(也就是线代中的单位化向量的方法),而这样就可以让波形中的偏振角还有视角与坐标系中一一对应。}
\paragraph{2.在尝试复现的过程,发现了引力波中关键和数学强关联的点主要有傅里叶变换(主要涉及广义相对论的时域上),线性方程求解,极化分解(转化接受到的信号为可以理解的信号)这三点。}
\section{参考文献}
\paragraph{LeCun, Y., Bengio, Y., & Hinton, G. (2015). Deep learning. Nature, 521(7553), 436-444.}
\paragraph{Goodfellow, I., Bengio, Y., Courville, A., & Bengio, Y. (2016). Deep Learning. MIT Press.}
\paragraph{Krizhevsky, A., Sutskever, I., & Hinton, G. E. (2012). Imagenet classification with deep convolutional neural networks. Advances in Neural Information Processing Systems, 25, 1097-1105.}
\paragraph{Kennedy, J.& Eberhart,R.(1995). Particle swarm optimization. Proceedings of IEEE International Conference on Neural Networks, 4, 1942-1948.}
\paragraph{Ligo Scientific Collaboration. (2016). GW150914: The first observation of gravitational waves from a binary black hole merger. Physical Review Letters, 116(6), 061102.}
\paragraph{Gonzalez, M. E. (2018). Digital signal processing. Springer International Publishing.}

\end{document}